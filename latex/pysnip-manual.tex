\documentclass{article}

\usepackage{pysnip}

\setlength\parindent{0pt}
\title{ PySnip Manual}
\author{Andrea Censi}

\newcommand{\cmd}[1]{\texttt{\textbackslash #1}}
\begin{document}
\maketitle

\section{Introduction}

PySnip is inspired by \texttt{python.sty} developed by James Brotchie.
The workflow is different and it allows heavier jobs which are processed
offline.


\section{Installation}

\begin{verbatim}
    python setup.py develop
\end{verbatim}

\section{Basic use}

Use the \cmd{py} command in your LaTeX file:
\begin{verbatim}
    \py{print('Hello world' + '!'*5)}
\end{verbatim}

Then compile as follows:
\begin{verbatim}
    pdflatex file.tex
    pysnip-make -c make
\end{verbatim}

This is the result:
\begin{quote}
    \py[myjob]{print('Hello world' + '!'*5)}
\end{quote}


\section{Advanced usage}

You can pass a job name to \cmd{py}:
\begin{verbatim}
    \py[myjob]{print('Hello world' + '!'*5)}
\end{verbatim}

Look in the \texttt{snippets/} directory. You will see the files:

\begin{description}
\item[\texttt{myjob.py}] The source that you specified.
\item[\texttt{myjob.tex}] The resulting latex source.
\item[\texttt{myjob.rc}] The exit value for the script (0=success).
\item[\texttt{myjob.pyo}]  The cached copy of the source.
\end{description}

\section{Even more advanced usage}

Re-run all the scripts:

\begin{verbatim}
    pysnip-make -c remake
\end{verbatim}

Re-run all the scripts starting with the string ``\texttt{myjob}'':

\begin{verbatim}
    pysnip-make -c "remake myjob*""
\end{verbatim}

Start a Compmake shell:

\begin{verbatim}
    pysnip-make 
\end{verbatim}

\end{document}
